\documentclass[a5paper]{article}
\usepackage{amssymb}
\usepackage{amstext}
\usepackage{amsmath}
\usepackage[default,osfigures,scale=0.95]{opensans}
\usepackage[margin=2cm]{geometry}
\setlength{\parindent}{0pt}
\setlength{\parskip}{1.2em}

\usepackage{booktabs}
\usepackage{array}
\newcolumntype{L}{>{$}l<{$}}
\usepackage[hidelinks]{hyperref}

\renewcommand{\t}{\tau}
\newcommand{\s}{\sigma}
\newcommand{\R}{\mathbb{R}}
\newcommand{\order}{\text{order}}

\newtheorem{lemma}{Lemma}

\begin{document}

\section*{Group algebra of the permutations of three things}

Finlay Thompson
\bigskip


The permutation group can be generated by two elements, $\t$ and $\s$,
of order two and three respectively. 
\begin{align}
    < \t,\s \mid \t^2 = 1,\s^3 = 1, \s\t = \t\s^2 >
\end{align}
These elements generate the six element finite group $S_3$, the
simplest non-Abelian group of permutations.

\begin{table}[h]
\begin{center}
\begin{tabular}{L|LLLLLL}
       & 1       & \t     & \t\s   & \t\s^2 & \s     & \s^2   \\
\hline\addlinespace
1      & 1       & \t     & \t\s   & \t\s^2 & \s     & \s^2   \\
\t     & \t      & 1      & \s     & \s^2   & \t\s   & \t\s^2 \\
\t\s   & \t\s    & \s^2   & 1      & \s     & \t\s^2 & \t     \\
\t\s^2 & \t\s^2  & \s     & \s^2   & 1      & \t     & \t\s   \\
\s     & \s      & \t\s^2 & \t     & \t\s   & \s^2   & 1      \\
\s^2   & \s^2    & \t\s   & \t\s^2 & \t     & 1      & \s     \\
\end{tabular}
\end{center}
\caption{Multiplication table for six elements permutation group on
three elements.}
\end{table}

The group algebra $\R[S_3]$ is a six dimensional, semisimple,
associative algebra, which by Weddeburn's structure theorems, means
that it can be decomposed into three parts, corresponding to the tree
conjugacy classes $\{1\},\{\t,\t\s,\t\s^2\},\{\s,\s^2\}$.
\begin{equation}
    \R[S_3] = \R \oplus \R \oplus M_2(\R)
\label{eq:decomposition}
\end{equation}
The decomposition \autoref{eq:decomposition} can be constructed using
multiplication by three central idempotents.
\begin{align}
    \pi_0 &= \frac{1}{6}(1+\t)(1+\s+\s^2) \\
    \pi_1 &= \frac{1}{6}(1-\t)(1+\s+\s^2) \\
    \pi_2 &= \frac{1}{3}(2-\s-\s^2)
\end{align}
These elements satisfy the equations $\pi_i^2=\pi_i$, $\pi_i\pi_j=0$
when $i\neq j$. The decomposition corresponds to the three irreducible
representations of $S_3$. 

The first projection $\pi_0$ maps is acted on uniformly as
the identity, $\rho_0(x)y = y$. This can be seen by, for example:
\begin{align*}
    \t \pi_0 &= \t\frac{1}{6}(1+\t)(1+\s+\s^2) \\
     &= \frac{1}{6}(1+\t)(1+\s+\s^2) \\
     &= \pi_0
\end{align*}
or
\begin{align*}
    \s \pi_0 &= \s\frac{1}{6}(1+\t)(1+\s+\s^2) \\
     &= \frac{1}{6}\s(1+\t)(1+\s+\s^2) \\
     &= \frac{1}{6}(\s+\s\t)(1+\s+\s^2) \\
     &= \frac{1}{6}(\s+\t\s^2)(1+\s+\s^2) \\
     &= \frac{1}{6}(\s(1+\s+\s^2) +\t\s^2(1+\s+\s^2))\\
     &= \frac{1}{6}((1+\s+\s^2) +\t(1+\s+\s^2))\\
     &= \frac{1}{6}(1+\t)(1+\s+\s^2) \\
     &= \pi_0
\end{align*}

The second representation is based on the sign function, that takes
each element to its sign, $\rho_1(x)y = (-1)^{(\order(x)-1)}y$. This
can be seen:
\begin{align*}
    \t \pi_1 &= \t\frac{1}{6}(1-\t)(1+\s+\s^2) \\
     &= \frac{1}{6}(\t-1)(1+\s+\s^2) \\
     &= \frac{-1}{6}(1-\t)(1+\s+\s^2) \\
     &= -\pi_1
\end{align*}
or
\begin{align*}
    \s \pi_1 &= \s\frac{1}{6}(1-\t)(1+\s+\s^2) \\
     &= \frac{1}{6}\s(1-\t)(1+\s+\s^2) \\
     &= \frac{1}{6}(\s-\s\t)(1+\s+\s^2) \\
     &= \frac{1}{6}(\s-\t\s^2)(1+\s+\s^2) \\
     &= \frac{1}{6}(\s(1+\s+\s^2) - \t\s^2(1+\s+\s^2))\\
     &= \frac{1}{6}((1+\s+\s^2) - \t(1+\s+\s^2))\\
     &= \frac{1}{6}(1-\t)(1+\s+\s^2) \\
     &= \pi_1
\end{align*}

The third representation can be understood by considering the two dimensional
subspace of $\{1,\s,\s^2\}_\R$ that is perpendicular to $1 + \s + \s^2$, with
the standard Euclidean metric. The element $\pi_2$ is part of this, and should
be associated to the identity element in $M_2(\R)$. Perpendicular to both
$\pi_2$ and $1+\s+\s^2$ is the element $\s-\s^2$. Considering the span of these
with $\t$ gets us the four dimensional space $M_2(\R)$.

\begin{align*}
\left(\begin{smallmatrix}1&0\\0&1\end{smallmatrix}\right) 
    \sim \frac{1}{3}(2 - \s - \s^2) &\quad
\left(\begin{smallmatrix}1&0\\0&-1\end{smallmatrix}\right) 
    \sim \frac{1}{\sqrt3}(\s - \s^2) \\
\left(\begin{smallmatrix}0&1\\1&0\end{smallmatrix}\right) 
    \sim \frac{\t}{3}(2 - \s - \s^2) &\quad
\left(\begin{smallmatrix}0&1\\-1&0\end{smallmatrix}\right) 
    \sim \frac{\t}{\sqrt3}(\s - \s^2)
\end{align*}

\begin{lemma}
    The isomorphism between $M_2(\R)$ and the image of $\pi_2$ contained in
    $\R[S_3]$ is given by:
    \begin{align*}
        E_{00} &\mapsto \frac{ 1}{6}\left(2 + (\sqrt 3 - 1)\s - (\sqrt 3 + 1) \s^2\right) \\
        E_{11} &\mapsto \frac{ 1}{6}\left(2 - (\sqrt 3 + 1)\s + (\sqrt 3 - 1) \s^2\right) \\
        E_{01} &\mapsto \frac{\t}{6}\left(2 + (\sqrt 3 - 1)\s - (\sqrt 3 + 1) \s^2\right) \\
        E_{10} &\mapsto \frac{\t}{6}\left(2 - (\sqrt 3 + 1)\s + (\sqrt 3 - 1) \s^2\right) 
    \end{align*}
\end{lemma}

\textbf{Proof:} I will just look at the just two maps to see that they work:



\end{document}
